%%%%%%%%%%%%%%%%%%%%%%%%%%%%%%%%%%%%%%%%%%%%%%%%%%%%%%%%%%%%%%%%%%%%%%%%%%%%%%%
%                         File: osa-revtex4-1.tex                             %
%                        Date: April 15, 2013                                 %
%                                                                             %
%                              BETA VERSION!                                  %
%                   JOSA A, JOSA B, Applied Optics, Optics Letters            %
%                                                                             %
%            This file requires the substyle file osajnl4-1.rtx,              %
%                   running under REVTeX 4.1 and LaTeX 2e                     %
%                                                                             %
%                   USE THE FOLLOWING REVTeX 4-1 OPTIONS:                     %
% \documentclass[osajnl,twocolumn,showpacs,superscriptaddress,10pt]{revtex4-1}%
%                    %% Use 11pt for Applied Optics                           %
%                                                                             %
%               (c) 2013 The Optical Society of America                       %
%                                                                             %
%%%%%%%%%%%%%%%%%%%%%%%%%%%%%%%%%%%%%%%%%%%%%%%%%%%%%%%%%%%%%%%%%%%%%%%%%%%%%%%

%%\documentclass[osajnl,twocolumn,showpacs,superscriptaddress,10pt]{revtex4-1} %% use 11pt for Applied Optics
\documentclass[osajnl,preprint,showpacs,superscriptaddress,10pt]{revtex4-1} %% use 12pt for preprint option
\usepackage{amsmath,nccmath,amssymb,graphicx,float,minted}
\usepackage[utf8]{inputenc}
\graphicspath{{images/}}

\usepackage{mathtools}
\DeclarePairedDelimiter\bra{\langle}{\rvert}
\DeclarePairedDelimiter\ket{\lvert}{\rangle}
\DeclarePairedDelimiterX\braket[2]{\langle}{\rangle}{#1 \delimsize\vert #2}

\setcounter{section}{-1}

\begin{document}

\title{Quantum Computing and Digital Physics}

\author{Ulises Jeremias Cornejo Fandos}
\affiliation{Licenciatura en Informática, Facultad de Informática, UNLP}

\begin{abstract}

\end{abstract}

\maketitle %% required

\section{Basic Operations}

\subsection{Inner/outer products in Dirac notation}

Sean $\ket{0} = \begin{pmatrix}1 & 0\end{pmatrix}^t$, $\ket{1} = \begin{pmatrix}0 & 1\end{pmatrix}^t$ y teniendo en cuenta $\braket{i}{j} = \delta_{ij}$, \\

Luego, por definición de transpuesta conjugada de una matriz, $\ket{0} = \begin{pmatrix}1 \\ 0\end{pmatrix}$, $\ket{1} = \begin{pmatrix}0 \\ 1\end{pmatrix}$, $\ket{0}^t = \begin{pmatrix}1 & 0\end{pmatrix}$, $\ket{1}^t = \begin{pmatrix}0 & 1\end{pmatrix}$ \\

\begin{itemize}

    \item $\begin{pmatrix}1 & 0\end{pmatrix}\begin{pmatrix}1 \\ 0\end{pmatrix} = \begin{pmatrix}1\end{pmatrix}$ \\
    
    En notación de Dirac, $\begin{pmatrix}1 & 0\end{pmatrix}\begin{pmatrix}1 \\ 0\end{pmatrix} = \ket{0}^t\ket{0} = \braket{0}{0} = 1$.
    
    \item $\begin{pmatrix}1 & 0\end{pmatrix}\begin{pmatrix}0 \\ 1\end{pmatrix} = \begin{pmatrix}0\end{pmatrix}$
    
    En notación de Dirac, $\begin{pmatrix}1 & 0\end{pmatrix}\begin{pmatrix}0 \\ 1\end{pmatrix} = \ket{0}^t\ket{1} = \braket{0}{1} = 0$.
    
    \item $\begin{pmatrix}1 & 2\end{pmatrix}\begin{pmatrix}3 \\ 4\end{pmatrix} = \begin{pmatrix}11\end{pmatrix}$ \\
    
    En notación de Dirac,
    \begin{fleqn}[\parindent]
    \begin{equation}
    \begin{split}
        \begin{pmatrix}1 & 2\end{pmatrix}\begin{pmatrix}3 \\ 4\end{pmatrix}
            &= (\bra{0} + 2\bra{1})(3\ket{0} + 4\ket{1}) \\
            &= 3\braket{0}{0} + 4\braket{0}{1} + 6\braket{1}{0} + 8\braket{1}{1} \\
            &= 3 + 8 \\
            &= 11
    \end{split}
    \end{equation}
    \end{fleqn}
    
\end{itemize}

Luego, pensando $\ket{i}\bra{j}$ como una matriz donde $(i, j) = 1$ y el resto todos en 0,

\begin{itemize}
    \item $\begin{pmatrix}1 \\ 0\end{pmatrix}\begin{pmatrix}1 & 0\end{pmatrix} = \begin{pmatrix}1 & 0 \\ 0 & 0\end{pmatrix}$ \\
    
    En notación de Dirac, $\begin{pmatrix}1 \\ 0\end{pmatrix}\begin{pmatrix}1 & 0\end{pmatrix} = \ket{0}\bra{0} = \begin{pmatrix}1 & 0 \\ 0 & 0\end{pmatrix}$.
    
    \item $\begin{pmatrix}0 \\ 1\end{pmatrix}\begin{pmatrix}1 & 0\end{pmatrix} = \begin{pmatrix}0 & 0 \\ 1 & 0\end{pmatrix}$ \\
    
    En notación de Dirac, $\begin{pmatrix}0 \\ 1\end{pmatrix}\begin{pmatrix}1 & 0\end{pmatrix} = \ket{1}\bra{0} = \begin{pmatrix}0 & 0 \\ 1 & 0\end{pmatrix}$.
    
    \item $\begin{pmatrix}3 \\ 4\end{pmatrix}\begin{pmatrix}1 & 2\end{pmatrix} = \begin{pmatrix}3 & 6 \\ 4 & 8\end{pmatrix}$ \\
    
    En notación de Dirac,
    \begin{fleqn}[\parindent]
    \begin{equation}
    \begin{split}
        \begin{pmatrix}3 \\ 4\end{pmatrix}\begin{pmatrix}1 & 2\end{pmatrix}
            &= (3\ket{0} + 4\ket{1})(\bra{0} + 2\bra{1}) \\
            &= 3\ket{0}\bra{0} + 6\ket{0}\bra{1} + 4\ket{1}\bra{0} + 8\ket{1}\bra{1} \\
            &= 3\begin{pmatrix}1 & 0 \\ 0 & 0\end{pmatrix} + 6\begin{pmatrix}0 & 1 \\ 0 & 0\end{pmatrix} + 4\begin{pmatrix}0 & 0 \\ 1 & 0\end{pmatrix} + 8\begin{pmatrix}0 & 0 \\ 0 & 1\end{pmatrix} \\
            &= \begin{pmatrix}3 & 0 \\ 0 & 0\end{pmatrix} + \begin{pmatrix}0 & 6 \\ 0 & 0\end{pmatrix} + \begin{pmatrix}0 & 0 \\ 4 & 0\end{pmatrix} + \begin{pmatrix}0 & 0 \\ 0 & 8\end{pmatrix} \\
            &= \begin{pmatrix}3 & 6 \\ 4 & 8\end{pmatrix}
    \end{split}
    \end{equation}
    \end{fleqn}
\end{itemize}

\subsection{Matrix products in Dirac notation}

Pensando $\ket{i}\bra{j}$ como una matriz donde $(i, j) = 1$ y el resto todos en 0,

\begin{itemize}
    \item $\begin{pmatrix}0 & 0 \\ 1 & 0\end{pmatrix}\begin{pmatrix}1 \\ 0\end{pmatrix} = \begin{pmatrix}0 \\ 1\end{pmatrix}$ \\
    
    En notación de Dirac, $\begin{pmatrix}0 & 0 \\ 1 & 0\end{pmatrix}\begin{pmatrix}1 \\ 0\end{pmatrix} = \ket{1}\bra{0}\ket{0} = \ket{1}\braket{0}{0} = \ket{1} = \begin{pmatrix}0 \\ 1\end{pmatrix}$.
    
    \item $\begin{pmatrix}0 & 0 \\ 1 & 0\end{pmatrix}\begin{pmatrix}0 \\ 1\end{pmatrix} = \begin{pmatrix}0 \\ 0\end{pmatrix}$ \\
    
    En notación de Dirac, $\begin{pmatrix}0 & 0 \\ 1 & 0\end{pmatrix}\begin{pmatrix}0 \\ 1\end{pmatrix} = \ket{1}\bra{0}\ket{1} = \ket{1}\braket{0}{1} = 0\ket{1} = \begin{pmatrix}0 \\ 0\end{pmatrix}$.
    
    \item $\begin{pmatrix}1 & 0 \\ 0 & 1\end{pmatrix}\begin{pmatrix}1 & 3\\ 2 & 4\end{pmatrix} = \begin{pmatrix}1 & 3\\ 2 & 4\end{pmatrix}$ \\
    
    En notación de Dirac,
    \begin{fleqn}[\parindent]
    \begin{equation}
    \begin{split}
        \begin{pmatrix}1 & 0 \\ 0 & 1\end{pmatrix}\begin{pmatrix}1 & 3\\ 2 & 4\end{pmatrix}
            &= (\ket{0}\bra{0} + \ket{1}\bra{1})(\ket{0}\bra{0} + 3\ket{0}\bra{1} + 2\ket{1}\bra{0} + 4\ket{1}\bra{1}) \\
            &= \ket{0}\braket{0}{0}\bra{0} + 3 \ket{0}\braket{0}{0}\bra{1} + 2\ket{0}\braket{0}{1}\bra{0} + 4 \ket{0}\braket{0}{1}\bra{1} \\
            &+ \ket{1}\braket{1}{0}\bra{0} + 3 \ket{1}\braket{1}{0}\bra{1} + 2\ket{1}\braket{1}{1}\bra{0} + 4 \ket{1}\braket{1}{1}\bra{1} \\
            &= \ket{0}\bra{0} + 3 \ket{0}\bra{1} + 2\ket{1}\bra{0} + 4 \ket{1}\bra{1} \\
            &= \begin{pmatrix}1 & 3\\ 2 & 4\end{pmatrix}
    \end{split}
    \end{equation}
    \end{fleqn}
    
    \item $\begin{pmatrix}0 & 0 \\ 0 & 1\end{pmatrix}\begin{pmatrix}0 & 0\\ 0 & 1\end{pmatrix} = \begin{pmatrix}0 & 0\\ 0 & 1\end{pmatrix}$ \\
    
    En notación de Dirac,
    \begin{fleqn}[\parindent]
    \begin{equation}
    \begin{split}
        \begin{pmatrix}0 & 0 \\ 0 & 1\end{pmatrix}\begin{pmatrix}0 & 0\\ 0 & 1\end{pmatrix}
            &= \ket{1}\bra{1}\ket{1}\bra{1} \\
            &= \ket{1}\braket{1}{1}\bra{1} \\
            &= \ket{1}\bra{1} \\
            &= \begin{pmatrix}0 & 0\\ 0 & 1\end{pmatrix}
    \end{split}
    \end{equation}
    \end{fleqn}
    
    \item $\begin{pmatrix}\frac{1}{\sqrt{2}} & \frac{1}{\sqrt{2}} \\ \frac{1}{\sqrt{2}} & -\frac{1}{\sqrt{2}}\end{pmatrix}\begin{pmatrix}\frac{1}{\sqrt{2}} & \frac{1}{\sqrt{2}} \\ \frac{1}{\sqrt{2}} & -\frac{1}{\sqrt{2}}\end{pmatrix} = \frac{1}{2}\begin{pmatrix}1 & 1 \\ 1 & -1\end{pmatrix}\begin{pmatrix}1 & 1\\ 1 & -1\end{pmatrix} = \frac{1}{2}\begin{pmatrix}2 & 0 \\ 0 & 2\end{pmatrix} = \begin{pmatrix}1 & 0 \\ 0 & 1\end{pmatrix}$ \\
    
    En notación de Dirac,
    \begin{fleqn}[\parindent]
    \begin{equation}
    \begin{split}
        \begin{pmatrix}\frac{1}{\sqrt{2}} & \frac{1}{\sqrt{2}} \\ \frac{1}{\sqrt{2}} & -\frac{1}{\sqrt{2}}\end{pmatrix}\begin{pmatrix}\frac{1}{\sqrt{2}} & \frac{1}{\sqrt{2}} \\ \frac{1}{\sqrt{2}} & -\frac{1}{\sqrt{2}}\end{pmatrix}
            &= \left (\frac{1}{\sqrt{2}}\ket{0}\bra{0} + \frac{1}{\sqrt{2}}\ket{0}\bra{1} + \frac{1}{\sqrt{2}}\ket{1}\bra{0} -\frac{1}{\sqrt{2}}\ket{1}\bra{1} \right) \\
            &. \left (\frac{1}{\sqrt{2}}\ket{0}\bra{0} + \frac{1}{\sqrt{2}}\ket{0}\bra{1} + \frac{1}{\sqrt{2}}\ket{1}\bra{0} -\frac{1}{\sqrt{2}}\ket{1}\bra{1} \right) \\
            &= \frac{1}{\sqrt{2}}\frac{1}{\sqrt{2}} \left (\ket{0}\bra{0} + \ket{0}\bra{1} + \ket{1}\bra{0} -\ket{1}\bra{1} \right) \left (\ket{0}\bra{0} + \ket{0}\bra{1} + \ket{1}\bra{0} -\ket{1}\bra{1} \right) \\
            &= \frac{1}{2} \left (\ket{0}\bra{0} + \ket{0}\bra{1} + \ket{1}\bra{0} -\ket{1}\bra{1} \right) \left (\ket{0}\bra{0} + \ket{0}\bra{1} + \ket{1}\bra{0} -\ket{1}\bra{1} \right) \\
            &= \frac{1}{2} \left (2\ket{0}\bra{0} + 2\ket{1}\bra{1} \right) \\
            &= \frac{1}{2} 2 \left (\ket{0}\bra{0} + \ket{1}\bra{1} \right) \\
            &= \begin{pmatrix}1 & 0\\ 0 & 1\end{pmatrix}
    \end{split}
    \end{equation}
    \end{fleqn}
\end{itemize}

\end{document}