%%%%%%%%%%%%%%%%%%%%%%%%%%%%%%%%%%%%%%%%%%%%%%%%%%%%%%%%%%%%%%%%%%%%%%%%%%%%%%%
%                         File: osa-revtex4-1.tex                             %
%                        Date: April 15, 2013                                 %
%                                                                             %
%                              BETA VERSION!                                  %
%                   JOSA A, JOSA B, Applied Optics, Optics Letters            %
%                                                                             %
%            This file requires the substyle file osajnl4-1.rtx,              %
%                   running under REVTeX 4.1 and LaTeX 2e                     %
%                                                                             %
%                   USE THE FOLLOWING REVTeX 4-1 OPTIONS:                     %
% \documentclass[osajnl,twocolumn,showpacs,superscriptaddress,10pt]{revtex4-1}%
%                    %% Use 11pt for Applied Optics                           %
%                                                                             %
%               (c) 2013 The Optical Society of America                       %
%                                                                             %
%%%%%%%%%%%%%%%%%%%%%%%%%%%%%%%%%%%%%%%%%%%%%%%%%%%%%%%%%%%%%%%%%%%%%%%%%%%%%%%

%%\documentclass[osajnl,twocolumn,showpacs,superscriptaddress,10pt]{revtex4-1} %% use 11pt for Applied Optics
\documentclass[osajnl,preprint,showpacs,superscriptaddress,10pt]{revtex4-1} %% use 12pt for preprint option
\usepackage{amsmath,nccmath,amssymb,graphicx,float,minted,xparse}
\usepackage[utf8]{inputenc}
\graphicspath{{images/}}

\usepackage{mathtools}
\DeclarePairedDelimiter\bra{\langle}{\rvert}
\DeclarePairedDelimiter\ket{\lvert}{\rangle}
\DeclarePairedDelimiterX\braket[2]{\langle}{\rangle}{#1 \delimsize\vert #2}

\NewDocumentCommand{\tens}{t_}
 {%
  \IfBooleanTF{#1}
   {\tensop}
   {\otimes}%
 }
\NewDocumentCommand{\tensop}{m}
 {%
  \mathbin{\mathop{\otimes}\displaylimits_{#1}}%
 }

\setcounter{section}{-1}

\begin{document}

\title{Quantum Computing and Digital Physics}

\author{Ulises Jeremias Cornejo Fandos}
\affiliation{Licenciatura en Informática, Facultad de Informática, UNLP}

\begin{abstract}

\end{abstract}

\maketitle %% required

\section{Basic Operations}

\subsection{Inner/outer products in Dirac notation}

Sean $\ket{0} = \begin{pmatrix}1 & 0\end{pmatrix}^t$, $\ket{1} = \begin{pmatrix}0 & 1\end{pmatrix}^t$ y teniendo en cuenta $\braket{i}{j} = \delta_{ij}$, \\

Luego, por definición de transpuesta conjugada de una matriz, $\ket{0} = \begin{pmatrix}1 \\ 0\end{pmatrix}$, $\ket{1} = \begin{pmatrix}0 \\ 1\end{pmatrix}$, $\ket{0}^t = \begin{pmatrix}1 & 0\end{pmatrix}$, $\ket{1}^t = \begin{pmatrix}0 & 1\end{pmatrix}$ \\

\begin{itemize}

    \item $\begin{pmatrix}1 & 0\end{pmatrix}\begin{pmatrix}1 \\ 0\end{pmatrix} = \begin{pmatrix}1\end{pmatrix}$ \\
    
    En notación de Dirac, $\begin{pmatrix}1 & 0\end{pmatrix}\begin{pmatrix}1 \\ 0\end{pmatrix} = \ket{0}^t\ket{0} = \braket{0}{0} = 1$.
    
    \item $\begin{pmatrix}1 & 0\end{pmatrix}\begin{pmatrix}0 \\ 1\end{pmatrix} = \begin{pmatrix}0\end{pmatrix}$
    
    En notación de Dirac, $\begin{pmatrix}1 & 0\end{pmatrix}\begin{pmatrix}0 \\ 1\end{pmatrix} = \ket{0}^t\ket{1} = \braket{0}{1} = 0$.
    
    \item $\begin{pmatrix}1 & 2\end{pmatrix}\begin{pmatrix}3 \\ 4\end{pmatrix} = \begin{pmatrix}11\end{pmatrix}$ \\
    
    En notación de Dirac,
    \begin{fleqn}[\parindent]
    \begin{equation}
    \begin{split}
        \begin{pmatrix}1 & 2\end{pmatrix}\begin{pmatrix}3 \\ 4\end{pmatrix}
            &= (\bra{0} + 2\bra{1})(3\ket{0} + 4\ket{1}) \\
            &= 3\braket{0}{0} + 4\braket{0}{1} + 6\braket{1}{0} + 8\braket{1}{1} \\
            &= 3 + 8 \\
            &= 11
    \end{split}
    \end{equation}
    \end{fleqn}
\end{itemize}

Luego, pensando $\ket{i}\bra{j}$ como una matriz donde $(i, j) = 1$ y el resto todos en 0,

\begin{itemize}
    \item $\begin{pmatrix}1 \\ 0\end{pmatrix}\begin{pmatrix}1 & 0\end{pmatrix} = \begin{pmatrix}1 & 0 \\ 0 & 0\end{pmatrix}$ \\
    
    En notación de Dirac, $\begin{pmatrix}1 \\ 0\end{pmatrix}\begin{pmatrix}1 & 0\end{pmatrix} = \ket{0}\bra{0} = \begin{pmatrix}1 & 0 \\ 0 & 0\end{pmatrix}$.
    
    \item $\begin{pmatrix}0 \\ 1\end{pmatrix}\begin{pmatrix}1 & 0\end{pmatrix} = \begin{pmatrix}0 & 0 \\ 1 & 0\end{pmatrix}$ \\
    
    En notación de Dirac, $\begin{pmatrix}0 \\ 1\end{pmatrix}\begin{pmatrix}1 & 0\end{pmatrix} = \ket{1}\bra{0} = \begin{pmatrix}0 & 0 \\ 1 & 0\end{pmatrix}$.
    
    \item $\begin{pmatrix}3 \\ 4\end{pmatrix}\begin{pmatrix}1 & 2\end{pmatrix} = \begin{pmatrix}3 & 6 \\ 4 & 8\end{pmatrix}$ \\
    
    En notación de Dirac,
    \begin{fleqn}[\parindent]
    \begin{equation}
    \begin{split}
        \begin{pmatrix}3 \\ 4\end{pmatrix}\begin{pmatrix}1 & 2\end{pmatrix}
            &= (3\ket{0} + 4\ket{1})(\bra{0} + 2\bra{1}) \\
            &= 3\ket{0}\bra{0} + 6\ket{0}\bra{1} + 4\ket{1}\bra{0} + 8\ket{1}\bra{1} \\
            &= 3\begin{pmatrix}1 & 0 \\ 0 & 0\end{pmatrix} + 6\begin{pmatrix}0 & 1 \\ 0 & 0\end{pmatrix} + 4\begin{pmatrix}0 & 0 \\ 1 & 0\end{pmatrix} + 8\begin{pmatrix}0 & 0 \\ 0 & 1\end{pmatrix} \\
            &= \begin{pmatrix}3 & 0 \\ 0 & 0\end{pmatrix} + \begin{pmatrix}0 & 6 \\ 0 & 0\end{pmatrix} + \begin{pmatrix}0 & 0 \\ 4 & 0\end{pmatrix} + \begin{pmatrix}0 & 0 \\ 0 & 8\end{pmatrix} \\
            &= \begin{pmatrix}3 & 6 \\ 4 & 8\end{pmatrix}
    \end{split}
    \end{equation}
    \end{fleqn}
\end{itemize}

\subsection{Matrix products in Dirac notation}

Pensando $\ket{i}\bra{j}$ como una matriz donde $(i, j) = 1$ y el resto todos en 0,

\begin{itemize}
    \item $\begin{pmatrix}0 & 0 \\ 1 & 0\end{pmatrix}\begin{pmatrix}1 \\ 0\end{pmatrix} = \begin{pmatrix}0 \\ 1\end{pmatrix}$ \\
    
    En notación de Dirac, $\begin{pmatrix}0 & 0 \\ 1 & 0\end{pmatrix}\begin{pmatrix}1 \\ 0\end{pmatrix} = \ket{1}\bra{0}\ket{0} = \ket{1}\braket{0}{0} = \ket{1} = \begin{pmatrix}0 \\ 1\end{pmatrix}$.
    
    \item $\begin{pmatrix}0 & 0 \\ 1 & 0\end{pmatrix}\begin{pmatrix}0 \\ 1\end{pmatrix} = \begin{pmatrix}0 \\ 0\end{pmatrix}$ \\
    
    En notación de Dirac, $\begin{pmatrix}0 & 0 \\ 1 & 0\end{pmatrix}\begin{pmatrix}0 \\ 1\end{pmatrix} = \ket{1}\bra{0}\ket{1} = \ket{1}\braket{0}{1} = 0\ket{1} = \begin{pmatrix}0 \\ 0\end{pmatrix}$.
    
    \item $\begin{pmatrix}1 & 0 \\ 0 & 1\end{pmatrix}\begin{pmatrix}1 & 3\\ 2 & 4\end{pmatrix} = \begin{pmatrix}1 & 3\\ 2 & 4\end{pmatrix}$ \\
    
    En notación de Dirac,
    \begin{fleqn}[\parindent]
    \begin{equation}
    \begin{split}
        \begin{pmatrix}1 & 0 \\ 0 & 1\end{pmatrix}\begin{pmatrix}1 & 3\\ 2 & 4\end{pmatrix}
            &= (\ket{0}\bra{0} + \ket{1}\bra{1})(\ket{0}\bra{0} + 3\ket{0}\bra{1} + 2\ket{1}\bra{0} + 4\ket{1}\bra{1}) \\
            &= \ket{0}\braket{0}{0}\bra{0} + 3 \ket{0}\braket{0}{0}\bra{1} + 2\ket{0}\braket{0}{1}\bra{0} + 4 \ket{0}\braket{0}{1}\bra{1} \\
            &+ \ket{1}\braket{1}{0}\bra{0} + 3 \ket{1}\braket{1}{0}\bra{1} + 2\ket{1}\braket{1}{1}\bra{0} + 4 \ket{1}\braket{1}{1}\bra{1} \\
            &= \ket{0}\bra{0} + 3 \ket{0}\bra{1} + 2\ket{1}\bra{0} + 4 \ket{1}\bra{1} \\
            &= \begin{pmatrix}1 & 3\\ 2 & 4\end{pmatrix}
    \end{split}
    \end{equation}
    \end{fleqn}
    
    \item $\begin{pmatrix}0 & 0 \\ 0 & 1\end{pmatrix}\begin{pmatrix}0 & 0\\ 0 & 1\end{pmatrix} = \begin{pmatrix}0 & 0\\ 0 & 1\end{pmatrix}$ \\
    
    En notación de Dirac,
    \begin{fleqn}[\parindent]
    \begin{equation}
    \begin{split}
        \begin{pmatrix}0 & 0 \\ 0 & 1\end{pmatrix}\begin{pmatrix}0 & 0\\ 0 & 1\end{pmatrix}
            &= \ket{1}\bra{1}\ket{1}\bra{1} \\
            &= \ket{1}\braket{1}{1}\bra{1} \\
            &= \ket{1}\bra{1} \\
            &= \begin{pmatrix}0 & 0\\ 0 & 1\end{pmatrix}
    \end{split}
    \end{equation}
    \end{fleqn}
    
    \item $\begin{pmatrix}\frac{1}{\sqrt{2}} & \frac{1}{\sqrt{2}} \\ \frac{1}{\sqrt{2}} & -\frac{1}{\sqrt{2}}\end{pmatrix}\begin{pmatrix}\frac{1}{\sqrt{2}} & \frac{1}{\sqrt{2}} \\ \frac{1}{\sqrt{2}} & -\frac{1}{\sqrt{2}}\end{pmatrix} = \frac{1}{2}\begin{pmatrix}1 & 1 \\ 1 & -1\end{pmatrix}\begin{pmatrix}1 & 1\\ 1 & -1\end{pmatrix} = \frac{1}{2}\begin{pmatrix}2 & 0 \\ 0 & 2\end{pmatrix} = \begin{pmatrix}1 & 0 \\ 0 & 1\end{pmatrix}$ \\
    
    En notación de Dirac,
    \begin{fleqn}[\parindent]
    \begin{equation}
    \begin{split}
        \begin{pmatrix}\frac{1}{\sqrt{2}} & \frac{1}{\sqrt{2}} \\ \frac{1}{\sqrt{2}} & -\frac{1}{\sqrt{2}}\end{pmatrix}\begin{pmatrix}\frac{1}{\sqrt{2}} & \frac{1}{\sqrt{2}} \\ \frac{1}{\sqrt{2}} & -\frac{1}{\sqrt{2}}\end{pmatrix}
            &= \left (\frac{1}{\sqrt{2}}\ket{0}\bra{0} + \frac{1}{\sqrt{2}}\ket{0}\bra{1} + \frac{1}{\sqrt{2}}\ket{1}\bra{0} -\frac{1}{\sqrt{2}}\ket{1}\bra{1} \right) \\
            &. \left (\frac{1}{\sqrt{2}}\ket{0}\bra{0} + \frac{1}{\sqrt{2}}\ket{0}\bra{1} + \frac{1}{\sqrt{2}}\ket{1}\bra{0} -\frac{1}{\sqrt{2}}\ket{1}\bra{1} \right) \\
            &= \frac{1}{\sqrt{2}}\frac{1}{\sqrt{2}} \left (\ket{0}\bra{0} + \ket{0}\bra{1} + \ket{1}\bra{0} -\ket{1}\bra{1} \right) \left (\ket{0}\bra{0} + \ket{0}\bra{1} + \ket{1}\bra{0} -\ket{1}\bra{1} \right) \\
            &= \frac{1}{2} \left (\ket{0}\bra{0} + \ket{0}\bra{1} + \ket{1}\bra{0} -\ket{1}\bra{1} \right) \left (\ket{0}\bra{0} + \ket{0}\bra{1} + \ket{1}\bra{0} -\ket{1}\bra{1} \right) \\
            &= \frac{1}{2} \left (2\ket{0}\bra{0} + 2\ket{1}\bra{1} \right) \\
            &= \frac{1}{2} 2 \left (\ket{0}\bra{0} + \ket{1}\bra{1} \right) \\
            &= \begin{pmatrix}1 & 0\\ 0 & 1\end{pmatrix}
    \end{split}
    \end{equation}
    \end{fleqn}
\end{itemize}

\subsection{Tensor products in Dirac notation}

Teniendo en cuenta que $\ket{i} \tens \ket{j} = \ket{i j}$ es un vector con un 1 en la posición "ij" y 0 en el resto, \\

\begin{itemize}
    \item $\begin{pmatrix}1 \\ 0\end{pmatrix} \tens \begin{pmatrix}0 \\ 1\end{pmatrix} = \begin{pmatrix}0 \\ 1 \\ 0 \\ 0\end{pmatrix}$ \\
    
    En notación de Dirac, $\begin{pmatrix}1 \\ 0\end{pmatrix} \tens \begin{pmatrix}0 \\ 1\end{pmatrix} = \ket{0} \tens \ket{1} = \ket{01} = \begin{pmatrix}0 \\ 1 \\ 0 \\ 0\end{pmatrix}$
    
    \item $\begin{pmatrix}0 \\ 1\end{pmatrix} \tens \begin{pmatrix}1 \\ 0\end{pmatrix} = \begin{pmatrix}0 \\ 0 \\ 1 \\ 0\end{pmatrix}$ \\
    
    En notación de Dirac, $\begin{pmatrix}0 \\ 1\end{pmatrix} \tens \begin{pmatrix}1 \\ 0\end{pmatrix} = \ket{1} \tens \ket{0} = \ket{10} = \begin{pmatrix}0 \\ 0 \\ 1 \\ 0\end{pmatrix}$
    
    \item $\begin{pmatrix}1 \\ 2\end{pmatrix} \tens \begin{pmatrix}3 \\ 4\end{pmatrix} = \begin{pmatrix}3 \\ 4 \\ 6 \\ 8\end{pmatrix}$ \\
    
    En notación de Dirac,
    \begin{fleqn}[\parindent]
    \begin{equation}
    \begin{split}
        \begin{pmatrix}1 \\ 2\end{pmatrix} \tens \begin{pmatrix}3 \\ 4\end{pmatrix}
            &= (\ket{0} + 2\ket{1}) \tens (3\ket{0} + 4\ket{1}) \\
            &= \ket{0} \tens 3\ket{0} + \ket{0} \tens 4\ket{1} + 2\ket{1} \tens 3\ket{0} + 2\ket{1} \tens 4\ket{1} \\
            &= 3\ket{00} + 4\ket{01} + 6\ket{10} + 8\ket{11} \\
            &= \begin{pmatrix}3 \\ 4 \\ 6 \\ 8\end{pmatrix}
    \end{split}
    \end{equation}
    \end{fleqn}
    
    \item $\begin{pmatrix}1 \\ 0\end{pmatrix} \tens \begin{pmatrix}1 \\ 0\end{pmatrix} + \begin{pmatrix}0 \\ 1\end{pmatrix} \tens \begin{pmatrix}0 \\ 1\end{pmatrix} = \begin{pmatrix}1 \\ 0 \\ 0 \\ 0\end{pmatrix} + \begin{pmatrix}0 \\ 0 \\ 0 \\ 1\end{pmatrix} = \begin{pmatrix}1 \\ 0 \\ 0 \\ 1\end{pmatrix}$ \\
    
    En notación de Dirac, $\begin{pmatrix}1 \\ 0\end{pmatrix} \tens \begin{pmatrix}1 \\ 0\end{pmatrix} + \begin{pmatrix}0 \\ 1\end{pmatrix} \tens \begin{pmatrix}0 \\ 1\end{pmatrix} = \ket{0} \tens \ket{0} + \ket{1} \tens \ket{1} = \ket{00} + \ket{11} = \begin{pmatrix}1 \\ 0 \\ 0 \\ 1\end{pmatrix}$
    
    \item $\begin{pmatrix}
        1 & 0 \\
        0 & 1
    \end{pmatrix}
    \tens 
    \begin{pmatrix}
        1  & 0 \\
        0 & 1
    \end{pmatrix} =
    \begin{pmatrix}
        1 & 0 & 0 & 0 \\
        0 & 1 & 0 & 0 \\
        0 & 0 & 1 & 0 \\
        0 & 0 & 0 & 1
    \end{pmatrix}$
    
    En notación de Dirac,
    \begin{fleqn}[\parindent]
    \begin{equation}
    \begin{split}
        \ket{0} \bra{0} + \ket{1} \bra{1}) \tens (\ket{0} \bra{0} + \ket{1} \bra{1})
        &= \ket{0} \bra{0} \tens \ket{0} \bra{0} + \ket{0} \bra{0} \tens \ket{1} \bra{1} + \ket{1} \bra{1} \tens \ket{0} \bra{0} + \ket{1} \bra{1} \tens \ket{1} \bra{1} \\
        &= \ket{00} \bra{00} + \ket{01} \bra{01} + \ket{10} \bra{10} + \ket{11} \bra{11} \\
        &= \begin{pmatrix}
            1 & 0 & 0 & 0 \\
            0 & 1 & 0 & 0 \\
            0 & 0 & 1 & 0 \\
            0 & 0 & 0 & 1
        \end{pmatrix}
    \end{split}
    \end{equation}
    \end{fleqn}
    
    \item $\begin{pmatrix}
        1 & 0 \\
        0 & 0
    \end{pmatrix}
    \tens 
    \begin{pmatrix}
        1  & 0 \\
        0 & 1
    \end{pmatrix} =
    \begin{pmatrix}
        1 & 0 & 0 & 0 \\
        0 & 1 & 0 & 0 \\
        0 & 0 & 0 & 0 \\
        0 & 0 & 0 & 0
    \end{pmatrix}$
    
    En notación de Dirac,
    \begin{fleqn}[\parindent]
    \begin{equation}
    \begin{split}
        \ket{0} \bra{0} \tens (\ket{0} \bra{0} + \ket{1} \bra{1})
        &= \ket{0} \bra{0} \tens \ket{0} \bra{0} + \ket{0} \bra{0} \tens \ket{1} \bra{1} \\
        &= \ket{00} \bra{00} + \ket{01} \bra{01} \\
        &= \begin{pmatrix}
            1 & 0 & 0 & 0 \\
            0 & 1 & 0 & 0 \\
            0 & 0 & 0 & 0 \\
            0 & 0 & 0 & 0
        \end{pmatrix}
    \end{split}
    \end{equation}
    \end{fleqn}
    
    \item $\begin{pmatrix}
        \frac{1}{\sqrt{2}} & \frac{1}{\sqrt{2}} \\
        \frac{1}{\sqrt{2}} & -\frac{1}{\sqrt{2}}
    \end{pmatrix}
    \tens 
    \begin{pmatrix}
        \frac{1}{\sqrt{2}} & \frac{1}{\sqrt{2}} \\
        \frac{1}{\sqrt{2}} & -\frac{1}{\sqrt{2}}
    \end{pmatrix} = \frac{1}{2}
    \begin{pmatrix}
        1 & 1 & 1 & 1 \\
        1 & -1 & 1 & -1 \\
        1 & 1 & -1 & -1 \\
        1 & -1 & -1 & 1
    \end{pmatrix}$
    
    En notación de Dirac,
    \begin{fleqn}[\parindent]
    \begin{equation}
    \begin{split}
        (\frac{1}{\sqrt{2}} \left (\ket{0}\bra{0} + \ket{0}\bra{1} + \ket{1}\bra{0} -\ket{1}\bra{1} \right)) 
        &\tens (\frac{1}{\sqrt{2}} \left (\ket{0}\bra{0} + \ket{0}\bra{1} + \ket{1}\bra{0} -\ket{1}\bra{1} \right)) \\
        &= \frac{1}{2} (\ket{0} \bra{0} \tens \ket{0} \bra{0} + \ket{0} \bra{0} \tens \ket{0} \bra{1} + \ket{0} \bra{0} \tens \ket{1} \bra{0} - \ket{0} \bra{0} \tens \ket{1} \bra{1} \\
        &+ \ket{0} \bra{1} \tens \ket{0} \bra{0} + \ket{0} \bra{1} \tens \ket{0} \bra{1} + \ket{0} \bra{1} \tens \ket{1} \bra{0} - \ket{0} \bra{1} \tens \ket{1} \bra{1} \\
        &+ \ket{1} \bra{0} \tens \ket{0} \bra{0} + \ket{1} \bra{0} \tens \ket{0} \bra{1} + \ket{1} \bra{0} \tens \ket{1} \bra{0} - \ket{1} \bra{0} \tens \ket{1} \bra{1} \\
        &- \ket{1} \bra{1} \tens \ket{0} \bra{0} - \ket{1} \bra{1} \tens \ket{0} \bra{1} - \ket{1} \bra{1} \tens \ket{1} \bra{0} + \ket{1} \bra{1} \tens \ket{1} \bra{1}) \\
        &= \frac{1}{2} \ket{00} \bra{00} + \ket{00} \bra{01} + \ket{01} \bra{00} - \ket{01} \bra{01} \\
        &+ \ket{00} \bra{10} + \ket{00} \bra{11} + \ket{01} \bra{10} - \ket{01} \bra{11} \\
        &+ \ket{10} \bra{00} + \ket{10} \bra{01} + \ket{11} \bra{00} - \ket{11} \bra{01} \\
        &- \ket{10} \bra{10} - \ket{10} \bra{11} - \ket{11} \bra{10} + \ket{11} \bra{11} \\
        &= \begin{pmatrix}
            1 & 1 & 1 & 1 \\
            1 & -1 & 1 & -1 \\
            1 & 1 & -1 & -1 \\
            1 & -1 & -1 & 1
        \end{pmatrix}
    \end{split}
    \end{equation}
    \end{fleqn}

\end{itemize}

\subsection{Dagger in Dirac notation}

\begin{itemize}
    \item $\begin{pmatrix}\frac{1}{\sqrt{2}} & \frac{1}{\sqrt{2}} \\ \frac{1}{\sqrt{2}} & -\frac{1}{\sqrt{2}}\end{pmatrix}^t = \begin{pmatrix}\frac{1}{\sqrt{2}} & \frac{1}{\sqrt{2}} \\ \frac{1}{\sqrt{2}} & -\frac{1}{\sqrt{2}}\end{pmatrix}$ \\
    
    En notación de Dirac,
    \begin{fleqn}[\parindent]
    \begin{equation}
    \begin{split}
        \begin{pmatrix}\frac{1}{\sqrt{2}} & \frac{1}{\sqrt{2}} \\ \frac{1}{\sqrt{2}} & -\frac{1}{\sqrt{2}}\end{pmatrix}^t
            &= \left (\frac{1}{\sqrt{2}}\begin{pmatrix}1 & 1 \\ 1 & -1\end{pmatrix} \right)^t \\
            &= \frac{1}{\sqrt{2}} \begin{pmatrix}1 & 1 \\ 1 & -1\end{pmatrix}^t \\
            &= \frac{1}{\sqrt{2}} \begin{pmatrix}1 & 1 \\ 1 & -1\end{pmatrix} \\
            &= \begin{pmatrix}\frac{1}{\sqrt{2}} & \frac{1}{\sqrt{2}} \\ \frac{1}{\sqrt{2}} & -\frac{1}{\sqrt{2}}\end{pmatrix}
    \end{split}
    \end{equation}
    \end{fleqn}
    
    \item $\begin{pmatrix}1 & 3i \\ 2 & 4i\end{pmatrix}^t = \begin{pmatrix}1 & 2 \\ -3i & -4i\end{pmatrix}$ \\
    
    En notación de Dirac,
    \begin{fleqn}[\parindent]
    \begin{equation}
    \begin{split}
        \begin{pmatrix}1 & 3i \\ 2 & 4i\end{pmatrix}^t
            &= \left (\ket{0}\bra{0} + 3i\ket{0}\bra{1} + 2\ket{1}\bra{0} + 4i\ket{1}\bra{1} \right)^t \\
            &=  (\ket{0}\bra{0})^t - 3i(\ket{0}{1})^t + 2(\ket{1}\bra{0})^t -4i(\ket{1}{1})^t \\
            &=  \ket{0}\bra{0} - 3i\ket{1}\bra{0} + 2\ket{0}\bra{1} - 4i\ket{1}{1} \\
            &=  \ket{0}\bra{0} + 2\ket{0}\bra{1} - 3i\ket{1}\bra{0} - 4i\ket{1}{1} \\
            &= \begin{pmatrix}1 & 2 \\ -3i & -4i\end{pmatrix}
    \end{split}
    \end{equation}
    \end{fleqn}
\end{itemize}

\subsection{Unitary gates in Dirac notation}

% H
$H = \begin{pmatrix}\frac{1}{\sqrt{2}} & \frac{1}{\sqrt{2}} \\ \frac{1}{\sqrt{2}} & -\frac{1}{\sqrt{2}}\end{pmatrix} = \frac{1}{\sqrt{2}} \left (\ket{0}\bra{0} + \ket{0}\bra{1} + \ket{1}\bra{0} -\ket{1}\bra{1} \right)$

$H^t = H$

\begin{fleqn}[\parindent]
\begin{equation}
\begin{split}
    H^t H
        &= (\frac{1}{\sqrt{2}} \left (\ket{0}\bra{0} + \ket{0}\bra{1} + \ket{1}\bra{0} -\ket{1}\bra{1} \right) (\frac{1}{\sqrt{2}} \left (\ket{0}\bra{0} + \ket{0}\bra{1} + \ket{1}\bra{0} -\ket{1}\bra{1} \right) \\
        &= \frac{1}{2} (\ket{0} \braket{0}{0} \bra{0} + \ket{0} \braket{0}{0} \bra{1} + \ket{0} \braket{0}{1} \bra{0} - \ket{0} \braket{0}{1} \bra{1} \\
        &+ \ket{0} \braket{1}{0} \bra{0} + \ket{0} \braket{1}{0} \bra{1} + \ket{0} \braket{1}{1} \bra{0} - \ket{0} \braket{1}{1} \bra{1} \\
        &+ \ket{1} \braket{0}{0} \bra{0} + \ket{1} \braket{0}{0} \bra{1} + \ket{1} \braket{0}{1} \bra{0} - \ket{1} \braket{0}{1} \bra{1} \\
        &- \ket{1} \braket{1}{0} \bra{0} - \ket{1} \braket{1}{0} \bra{1} - \ket{1} \braket{1}{1} \bra{0} + \ket{1} \braket{1}{1} \bra{1}) \\
        &= \frac{1}{2} (\ket{0} \bra{0} + \ket{0} \bra{1} + \ket{0} \bra{0} - \ket{0} \bra{1} + \ket{1} \bra{0} + \ket{1} \bra{1} - \ket{1} \bra{0} + \ket{1} \bra{1}) \\
        &= \frac{1}{2} 2 (\ket{0} \bra{0} + \ket{1} \bra{1}) = \ket{0} \bra{0} + \ket{1} \bra{1} \\
        &= I
\end{split}
\end{equation}
\end{fleqn}

% CNOT

$CNOT =
\begin{pmatrix}
    1 & 0 & 0 & 0 \\
    0 & 1 & 0 & 0 \\
    0 & 0 & 0 & 1 \\
    0 & 0 & 1 & 0
\end{pmatrix} = \ket{00} \bra{00} + \ket{01} \bra{01} + \ket{10} \bra{11} + \ket{11} \bra{10}$

$CNOT^t = CNOT$

\begin{fleqn}[\parindent]
\begin{equation}
\begin{split}
    CNOT^t CNOT
        &= (\ket{00} \bra{00} + \ket{01}) \bra{01} + \ket{10} \bra{11} + \ket{11} \bra{10}) (\ket{00} \bra{00} + \ket{01}) \bra{01} + \ket{10} \bra{11} + \ket{11} \bra{10}) \\
        &= \ket{00}\braket{00}{00}\bra{00} + \ket{00}\braket{00}{01}\bra{01} + \ket{00}\braket{00}{10}\bra{11} + \ket{00}\braket{00}{11}\bra{10} \\
        &+ \ket{01}\braket{01}{00}\bra{00} + \ket{01}\braket{01}{01}\bra{01} + \ket{01}\braket{01}{10}\bra{11} + \ket{01}\braket{01}{11}\bra{10} \\
        &+  \ket{10}\braket{11}{00}\bra{00} + \ket{10}\braket{11}{01}\bra{01} + \ket{10}\braket{11}{10}\bra{11} + \ket{10}\braket{11}{11}\bra{10} \\
        &+ \ket{11}\braket{10}{00}\bra{00} + \ket{11}\braket{10}{01}\bra{01} + \ket{11}\braket{10}{10}\bra{11} + \ket{11}\braket{10}{11}\bra{10} \\
        &= \ket{00} \bra{00} + \ket{01}\bra{01} + \ket{10} \bra{10} + \ket{11} \bra{11} \\
        &= I
\end{split}
\end{equation}
\end{fleqn}

% T

$T = \begin{pmatrix}1 & 0 \\ 0 & e^{\frac{i \pi}{4}}\end{pmatrix} = \ket{0} \bra{0} + e^{\frac{i \pi}{4}} \ket{1} \bra{1}$

$T^t = \ket{0} \bra{0} + e^{-\frac{i \pi}{4}} \ket{1} \bra{1}$


\begin{fleqn}[\parindent]
\begin{equation}
\begin{split}
    T^t T 
        &= (\ket{0} \bra{0} + e^{-\frac{i \pi}{4}} \ket{1} \bra{1}) \ket{0} \bra{0} + e^{\frac{i \pi}{4}} \ket{1} \bra{1}) \\
        &= \ket{0} \braket{0}{0} \bra{0} + e^{\frac{i \pi}{4}} \ket{0} \braket{0}{1} \bra{1} + e^{-\frac{i \pi}{4}} \ket{1} \braket{1}{0} + e^0 \ket{1} \braket{1}{1} \bra{1} \\
        &= \ket{0} \bra{0} + \ket{1} \bra{1} = I
\end{split}
\end{equation}
\end{fleqn}

\subsection{Evolutions}

Se debe aplicar $CNOT (H \tens I) \psi$.

\begin{fleqn}[\parindent]
\begin{equation}
\begin{split}
    (H \tens I) (\ket{0} \tens \ket{0})
        &= \begin{pmatrix}\frac{1}{\sqrt{2}} \begin{pmatrix} 1 & 0 \\ 0 & 1 \end{pmatrix}
            & \frac{1}{\sqrt{2}} \begin{pmatrix} 1 & 0 \\ 0 & 1 \end{pmatrix}
            \\ \frac{1}{\sqrt{2}} \begin{pmatrix} 1 & 0 \\ 0 & 1 \end{pmatrix}
            & -\frac{1}{\sqrt{2}} \begin{pmatrix} 1 & 0 \\ 0 & 1 \end{pmatrix} \end{pmatrix} \ket{00} \\
        &= \frac{1}{\sqrt{2}}
            \begin{pmatrix}
            1 & 0 & 1 & 0 \\
            0 & 1 & 0 & 1 \\
            1 & 0 & -1 & 0 \\
            0 & 1 & 0 & -1
        \end{pmatrix} \begin{pmatrix} 1 \\ 0 \\ 0 \\ 0 \end{pmatrix} \\
        &= \frac{1}{\sqrt{2}} \begin{pmatrix} 1 \\ 0 \\ 1 \\ 0 \end{pmatrix} \\
        &= \frac{\ket{00} + \ket{10}}{\sqrt{2}}
\end{split}
\end{equation}
\end{fleqn}

Se calcula CNOT sobre el resultado
\begin{fleqn}[\parindent]
\begin{equation}
\begin{split}
    CNOT(\frac{\ket{00} + \ket{10}}{\sqrt{2}})
        &= \frac{CNOT(\ket{00}) + CNOT(\ket{10})}{\sqrt{2}} \\
        &= \frac{\ket{00} + \ket{11}}{\sqrt{2}}
\end{split}
\end{equation}
\end{fleqn}

Aplicando el cambio nuevamente pero esta vez considerando la evolución  $(H \tens I) CNOT \psi'$. Primero calculamos CNOT sobre $\psi'$:

$CNOT(\frac{\ket{00} + \ket{10}}{\sqrt{2}}) = \frac{CNOT(\ket{00}) + CNOT(\ket{10})}{\sqrt{2}} = \frac{\ket{00} + \ket{11}}{\sqrt{2}}$

Realizamos el resto del cálculo para obtener $\psi''$:

\begin{fleqn}[\parindent]
\begin{equation}
\begin{split}
    (H \tens I)(\frac{\ket{00} + \ket{11}}{\sqrt{2}})
        &= \frac{1}{\sqrt{2}}
            \begin{pmatrix}
                1 & 0 & 1 & 0 \\
                0 & 1 & 0 & 1 \\
                1 & 0 & -1 & 0 \\
                0 & 1 & 0 & -1 
            \end{pmatrix} \frac{\ket{00} + \ket{11}}{\sqrt{2}} \\
        &= \frac{1}{\sqrt{2}}
            \begin{pmatrix}
                1 & 0 & 1 & 0 \\
                0 & 1 & 0 & 1 \\
                1 & 0 & -1 & 0 \\
                0 & 1 & 0 & -1 
            \end{pmatrix} \frac{1}{\sqrt{2}} \begin{pmatrix} 1 \\ 0 \\ 0 \\ 1 \end{pmatrix} \\
        &= \frac{1}{2} \begin{pmatrix} 1 \\ 1 \\ 1 \\ -1 \end{pmatrix} = \begin{pmatrix} \frac{1}{2} \\ \frac{1}{2} \\ \frac{1}{2} \\ -\frac{1}{2} \end{pmatrix} \\
        &= \frac{\ket{00} + \ket{01} + \ket{10} + \ket{11}}{2}
\end{split}
\end{equation}
\end{fleqn}

\subsection{Measuring in another basis}

\subsection{Measuring the phase}

\subsection{Measuring a subsystem}

Sean $\mathcal{P}_0 = \ket{0}\bra{0}$, $\mathcal{P}_1 = \mathcal{I} - \mathcal{P}_0$ y sabiendo que $\mathcal{P}$ es un proyector si y solo si $\mathcal{P}^2 = \mathcal{P}$: \\

$\mathcal{P}^2_0 = \ket{0}\bra{0}^2 = \ket{0}\braket{0}{0}\ket{0} = \ket{0}\bra{0} = \mathcal{P}_0$ [Es un proyector] \\

$\mathcal{P}^2_1 = (\mathcal{I} - \mathcal{P}_0)(\mathcal{I} - \mathcal{P}_0) = \mathcal{I} - \mathcal{P}_0 - \mathcal{P}_0 + \mathcal{P}^2_0 = \mathcal{I} - \mathcal{P}_0 - \mathcal{P}_0 + \mathcal{P}_0 = \mathcal{I} - \mathcal{P}_0 = \mathcal{P}_1$ [Es un proyector] \\

Sea $\mathcal{P}$ un proyector tenemos que $\mathcal{P} \tens \mathcal{I}$ es un proyector: \\

$(\mathcal{P} \tens \mathcal{I})^2 = \mathcal{P} \tens \mathcal{I})(\mathcal{P} \tens \mathcal{I}) = \mathcal{P}^2 \tens \mathcal{I} = \mathcal{P} \tens \mathcal{I}$ [Es un proyector] \\

La suma de ambos proyectores debe ser $\mathcal{I}$: \\

$\mathcal{P}_0 \tens \mathcal{I} + \mathcal{P}_1 \tens \mathcal{I} = \mathcal{P}_0 \tens \mathcal{I} + (\mathcal{I} - \mathcal{P}_0) \tens \mathcal{I} = (\mathcal{I} - \mathcal{P}_0 + \mathcal{P}_0) \tens \mathcal{I} = \mathcal{I} \tens \mathcal{I} = \mathcal{I}$ \\

Luego, calculamos el conjunto de salidas clásicas con sus probabilidades y el estado posterior a la medida:

\begin{itemize}
    \item Probabilidades
    
    \begin{itemize}
        \item Probabilidad de medir 0
        
        \begin{fleqn}[\parindent]
        \begin{equation}
        \begin{split}
            p(0)
                &= || (\mathcal{P}_0 \tens \mathcal{I}) \frac{\ket{00} + \ket{11}}{\sqrt{2}} ||^2 \\
                &= || \ket{00}\bra{00} + \ket{01}\bra{01} \frac{\ket{00} + \ket{11}}{\sqrt{2}} \\
                &= || \frac{\ket{11}}{\sqrt{2}} ||^2 \\
                &= \frac{1}{2}
        \end{split}
        \end{equation}
        \end{fleqn}
        
        \item Probabilidad de medir 1
        
        \begin{fleqn}[\parindent]
        \begin{equation}
        \begin{split}
            p(1)
                &= || (\mathcal{P}_1 \tens \mathcal{I}) \frac{\ket{00} + \ket{11}}{\sqrt{2}} ||^2 \\
                &= || \ket{10}\bra{10} + \ket{11}\bra{11} \frac{\ket{00} + \ket{11}}{\sqrt{2}} \\
                &= || \frac{\ket{00}}{\sqrt{2}} ||^2 \\
                &= \frac{1}{2}
        \end{split}
        \end{equation}
        \end{fleqn}
    \end{itemize}
    
    \item Estados posteriores
    
    \begin{itemize}
        \item \begin{fleqn}[\parindent]
        \begin{equation}
        \begin{split}
            \ket{\psi}_0
                &= \frac{(\mathcal{P}_0 \tens \mathcal{I}) \frac{\ket{00} + \ket{11}}{\sqrt{2}}}{\sqrt{\frac{1}{2}}} \\
                &= \frac{\ket{00}\sqrt{2}}{\sqrt{2}} \\
                &= \ket{00}
        \end{split}
        \end{equation}
        \end{fleqn}
        
        \item \begin{fleqn}[\parindent]
        \begin{equation}
        \begin{split}
            \ket{\psi}_1
                &= \frac{(\mathcal{P}_1 \tens \mathcal{I}) \frac{\ket{00} + \ket{11}}{\sqrt{2}}}{\sqrt{\frac{1}{2}}} \\
                &= \frac{\ket{11}\sqrt{2}}{\sqrt{2}} \\
                &= \ket{11}
        \end{split}
        \end{equation}
        \end{fleqn}
    \end{itemize}
\end{itemize}

\subsection{No-cloning}

¿Existe $\mathbf{U} (\forall \ket{\psi} \in \mathbf{C}^2): \ket{\psi} \tens \ket{0} = \ket{\psi} \tens \ket{\psi}$? \\

\begin{itemize}
    \item Probamos para los estados básicos,
    
    \begin{itemize}
        \item $\mathbf{U} \ket{0} \tens \ket{0} = \ket{0} \tens \ket{0}$, se cumple para $\ket{0}$
        \item $\mathbf{U} \ket{1} \tens \ket{0} = \ket{1} \tens \ket{1}$, se cumple para $\ket{1}$
    \end{itemize}
    
    \item Dado un estado superpuesto $\alpha \ket{0} + \beta \ket{1} \in \mathbf{C}^2$,
    
    \begin{fleqn}[\parindent]
    \begin{equation}
    \begin{split}
        \mathbf{U} (\alpha \ket{0} + \beta \ket{1}) \tens \ket{0}
            &= \alpha \mathbf{U} \ket{00} + \beta \mathbf{U} \ket{10} \\
            &= \alpha \ket{00} + \beta \ket{11} \\
            & \neq (\alpha \ket{0} + \beta \ket{1}) \tens (\alpha \ket{0} + \beta \ket{1}) \\
            &= \alpha^2 \ket{00} + \alpha \beta \ket{01} + \alpha \beta \ket{10} + \beta^2 \ket{11}
    \end{split}
    \end{equation}
    \end{fleqn}
\end{itemize}

Son distintos, por lo tanto no se puede clonar.

\section{Computations}

\subsection{Hadamard}

Sea $a \in \{0, 1\}$. Muestro que $H\ket{a} = \frac{(\ket{0} + (-1)^a \ket{1})}{\sqrt{2}}$. \\

$H\ket{0} = \ket{+} = \frac{\ket{0} + \ket{1}}{\sqrt{2}} = \frac{\ket{0} + (-1)^0\ket{1}}{\sqrt{2}}$ \\
    
$H\ket{1} = \ket{-} = \frac{\ket{0} - \ket{1}}{\sqrt{2}} = \frac{\ket{0} + (-1)^1\ket{1}}{\sqrt{2}}$

\subsection{Who controls whom?}

\section{Bell basis}

\section{Superdense-conding}

\section{Teleportation}

\section{Quantum key distribution}

\subsection{Canonical basis versus diagonal basis}

Si Alicia envía un bit de información en una base y Bob lo mide en la misma base mide con seguridad el mismo estado por lo que aprende sobre el bit original. En caso de que Bob midiera en una base distinta la transmisión de información sería nula dado que Bob no aprendería nada del bit original.

Si Alicia envia un bit de información en una de las bases, Eva lo intercepta y lo mide en la misma base y luego Bob lo mide nuevamente en la misma base, Bob aprende del bit original con un 100\% de probabilidad. A su vez, si Eva hubiese medido en otra base, las acciones de Eva pasan desapercibidas dado que Bob mide en la base original aprendiendo sobre el bit original.

\subsection{BB84}



\end{document}